\documentclass[11pt]{article}

% To produce a letter size output. Otherwise will be A4 size.
\usepackage[letterpaper]{geometry}

% For enumerated lists using letters: a. b. etc.
\usepackage{enumitem}

\topmargin -.5in
\textheight 9in
\oddsidemargin -.25in
\evensidemargin -.25in
\textwidth 7in

\begin{document}

% Edit the following putting your first and last names and your lab section.
\author{Anthony Sainez\\
Lab CSE-015-03L W 10:30-1:20pm}

% Edit the following replacing X with the HW number.
\title{CSE 015: Discrete Mathematics\\
Fall 2019\\
Homework \#2\\
Solution}

% Put today's date in the following.
\date{October 11, 2019}
\maketitle

% ========== Begin questions here
\begin{enumerate}

\item
\textbf{Question 1:}

\begin{enumerate}[label=(\alph*)]
\item
For all animals such that if it is a rabbit, then it hops.

\item
For all animals, given it is a rabbit, and it hops.

\item
There exists an animal, such that if it is a rabbit, then it hops.

\item
There exists an animal, such that it is a rabbit, and x hops.

\end{enumerate}

\item
\textbf{Question 2:}

\begin{enumerate}[label=(\alph*)]
\item
True, since $\sqrt{2}$ is a real number.

\item
False, since $i=\sqrt{-1}$ is not real.

\item
True, since $x^2 \geq 0$

\item
False. Take the counter example $x=0$ and $x=1$

\end{enumerate}

\item
\textbf{Question 3:}

In each case, $Q(x)$ is defined as "x is a student in your class" 

\begin{enumerate}[label=(\alph*)]
\item 
$\forall x P(x)$ and $\exists x (Q(x) \land P(x))$
\\ where $P(x)$ is "x has a cell phone."

\item
$\exists x P(x)$ and $\exists x (Q(x) \land P(x))$
\\ where $P(x)$ is "x has seen a foreign movie."

\item
$\exists x \lnot P(x)$ and $\exists x (Q(x) \land \lnot P(x))$
\\ where $P(x)$ is "x can swim."

\item
$\forall x P(x)$ and $\exists x (Q(x) \land P(x))$
\\ where $P(x)$ is "x can solve quadratic equations."

\end{enumerate}

\item
\textbf{Question 4:}

\begin{enumerate}[label=(\alph*)]
\item
There exists a student in your class who has taken a computer science course. 

\item
Every student in your class has taken at least one computer science course.

\item
Every computer science course has been taken by some student in your class. Alternatively, you can say that there is at least one student in your class who has taken all the computer science courses at your school. 
\end{enumerate}

\item
\textbf{Question 5:}

\begin{enumerate}[label=(\alph*)]
\item
$\exists x \exists y Q(x, y)$

\item
$\forall x \forall y \lnot Q(x, y)$

\end{enumerate}

\item
\textbf{Question 6:}

\begin{enumerate}[label=(\alph*)]
\item
$\forall x \exists y F(x, y)$

\item
$\lnot \exists x \forall y F(x, y)$ or $\forall x \exists y \lnot F(x, y)$

\item
$\exists x \forall y  F(x, y)$

\end{enumerate}

\item
\textbf{Question 7:}

\begin{enumerate}[label=(\alph*)]
\item
$\forall x \forall y [(x < 0 \land y < 0) \rightarrow xy > 0]$

\item
$\forall x \forall y [(x > 0 \land y > 0) \rightarrow \frac{x + y}{2} > 0]$

\item
$\forall x \forall y (\left| x + y \right| \leq \left| x \right| + \left| y \right|) $

\end{enumerate}

\item
\textbf{Question 8:}

\begin{enumerate}[label=(\alph*)]
\item
False, because if $x=-1$, then there does not exist a real number $y$ such that $x=y^2$. Recall that $\sqrt{-1}$ is not a real number.

\item
True. For every real number there exists another real number such that their sum equals $1$. There is no counter-example. 

\item
False. The statement essentially asks that we find the two real numbers that solve the system of equations $x + 2y = 2$ and $2x +4y = 5$. In slope intercept form, these lines can be written as $y=-\frac{1}{2}x + 1$ and $y=-\frac{1}{2}x + \frac{5}{4}$. These lines have the same slope and will never intercept, so they will never share an x and a y, and the system of equations does not have a solution. 

\end{enumerate}

\end{enumerate}

\end{document}
